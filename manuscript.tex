% Options for packages loaded elsewhere
\PassOptionsToPackage{unicode}{hyperref}
\PassOptionsToPackage{hyphens}{url}
\PassOptionsToPackage{dvipsnames,svgnames,x11names}{xcolor}
%
\documentclass[
]{article}
\usepackage{amsmath,amssymb}
\usepackage{iftex}
\ifPDFTeX
  \usepackage[T1]{fontenc}
  \usepackage[utf8]{inputenc}
  \usepackage{textcomp} % provide euro and other symbols
\else % if luatex or xetex
  \usepackage{unicode-math} % this also loads fontspec
  \defaultfontfeatures{Scale=MatchLowercase}
  \defaultfontfeatures[\rmfamily]{Ligatures=TeX,Scale=1}
\fi
\usepackage{lmodern}
\ifPDFTeX\else
  % xetex/luatex font selection
\fi
% Use upquote if available, for straight quotes in verbatim environments
\IfFileExists{upquote.sty}{\usepackage{upquote}}{}
\IfFileExists{microtype.sty}{% use microtype if available
  \usepackage[]{microtype}
  \UseMicrotypeSet[protrusion]{basicmath} % disable protrusion for tt fonts
}{}
\makeatletter
\@ifundefined{KOMAClassName}{% if non-KOMA class
  \IfFileExists{parskip.sty}{%
    \usepackage{parskip}
  }{% else
    \setlength{\parindent}{0pt}
    \setlength{\parskip}{6pt plus 2pt minus 1pt}}
}{% if KOMA class
  \KOMAoptions{parskip=half}}
\makeatother
\usepackage{xcolor}
\usepackage[margin=1in]{geometry}
\usepackage{longtable,booktabs,array}
\usepackage{calc} % for calculating minipage widths
% Correct order of tables after \paragraph or \subparagraph
\usepackage{etoolbox}
\makeatletter
\patchcmd\longtable{\par}{\if@noskipsec\mbox{}\fi\par}{}{}
\makeatother
% Allow footnotes in longtable head/foot
\IfFileExists{footnotehyper.sty}{\usepackage{footnotehyper}}{\usepackage{footnote}}
\makesavenoteenv{longtable}
\usepackage{graphicx}
\makeatletter
\def\maxwidth{\ifdim\Gin@nat@width>\linewidth\linewidth\else\Gin@nat@width\fi}
\def\maxheight{\ifdim\Gin@nat@height>\textheight\textheight\else\Gin@nat@height\fi}
\makeatother
% Scale images if necessary, so that they will not overflow the page
% margins by default, and it is still possible to overwrite the defaults
% using explicit options in \includegraphics[width, height, ...]{}
\setkeys{Gin}{width=\maxwidth,height=\maxheight,keepaspectratio}
% Set default figure placement to htbp
\makeatletter
\def\fps@figure{htbp}
\makeatother
\setlength{\emergencystretch}{3em} % prevent overfull lines
\providecommand{\tightlist}{%
  \setlength{\itemsep}{0pt}\setlength{\parskip}{0pt}}
\setcounter{secnumdepth}{5}
\newlength{\cslhangindent}
\setlength{\cslhangindent}{1.5em}
\newlength{\csllabelwidth}
\setlength{\csllabelwidth}{3em}
\newlength{\cslentryspacingunit} % times entry-spacing
\setlength{\cslentryspacingunit}{\parskip}
\newenvironment{CSLReferences}[2] % #1 hanging-ident, #2 entry spacing
 {% don't indent paragraphs
  \setlength{\parindent}{0pt}
  % turn on hanging indent if param 1 is 1
  \ifodd #1
  \let\oldpar\par
  \def\par{\hangindent=\cslhangindent\oldpar}
  \fi
  % set entry spacing
  \setlength{\parskip}{#2\cslentryspacingunit}
 }%
 {}
\usepackage{calc}
\newcommand{\CSLBlock}[1]{#1\hfill\break}
\newcommand{\CSLLeftMargin}[1]{\parbox[t]{\csllabelwidth}{#1}}
\newcommand{\CSLRightInline}[1]{\parbox[t]{\linewidth - \csllabelwidth}{#1}\break}
\newcommand{\CSLIndent}[1]{\hspace{\cslhangindent}#1}
\usepackage{caption}
\usepackage{booktabs}
\usepackage{longtable}
\usepackage{array}
\usepackage{multirow}
\usepackage{wrapfig}
\usepackage{float}
\usepackage{colortbl}
\usepackage{pdflscape}
\usepackage{tabu}
\usepackage{threeparttable}
\usepackage{threeparttablex}
\usepackage[normalem]{ulem}
\usepackage{makecell}
\usepackage{xcolor}
\ifLuaTeX
  \usepackage{selnolig}  % disable illegal ligatures
\fi
\IfFileExists{bookmark.sty}{\usepackage{bookmark}}{\usepackage{hyperref}}
\IfFileExists{xurl.sty}{\usepackage{xurl}}{} % add URL line breaks if available
\urlstyle{same}
\hypersetup{
  pdftitle={Parental Mentalization Across Cultures: Mind-mindedness and Parental Reflective Functioning in British and South Korean Mothers},
  pdfauthor={Yu Jin Lee, Elizabeth Meins \& Fionnuala Larkin},
  colorlinks=true,
  linkcolor={black},
  filecolor={Maroon},
  citecolor={Blue},
  urlcolor={Blue},
  pdfcreator={LaTeX via pandoc}}

\title{Parental Mentalization Across Cultures: Mind-mindedness and Parental Reflective Functioning in British and South Korean Mothers}
\author{Yu Jin Lee, Elizabeth Meins \& Fionnuala Larkin}
\date{}

\begin{document}
\maketitle

\hypertarget{abstract}{%
\section*{Abstract}\label{abstract}}
\addcontentsline{toc}{section}{Abstract}

Differences in mind-mindedness and Parental Reflective Functioning (PRF) were investigated in mothers and their 6-month-old infants from South Korea (N=66, 32 girls) and the United Kingdom (N=63, 26 girls). Mind-mindedness was assessed in terms of appropriate and non-attuned mind-related comments during infant--mother interaction; PRF was assessed using a questionnaire. British mothers commented more on infant desires and preferences, whereas Korean mothers commented more on cognitions and emotions, but there were no cultural differences in overall levels of mind-mindedness. For PRF, Korean mothers reported more certainty about their infants' mental states compared with their British counterparts, but there were no cultural differences in mothers' reported interest in their infants' mental states. Greater reported certainty about infants' mental states was positively related to self-reported parenting quality in both cultural groups, but this association was not seen for parenting quality as assessed observationally. Mind-mindedness and PRF were unrelated in both Korean and British mothers. Results are discussed in terms of the Korean concept of mother---infant \emph{oneness} and the multi-dimensional nature of parental mentalization.

\emph{Keywords}: Cultural differences, Maternal mind-mindedness, Parental Reflective Functioning, Parenting, Maternal well-being

\hypertarget{public-significance-statement}{%
\section*{Public significance statement}\label{public-significance-statement}}
\addcontentsline{toc}{section}{Public significance statement}

This study compared Korean and British mothers' understanding of their infants' minds (i.e., parental mentalization). The results suggest that parental mentalization is universal, but the specific way in which mothers engage with their infants' internal states differs according to their cultural context. Our findings also support conceptualizing parental mentalization as a multi-dimensional construct.

\hypertarget{introduction}{%
\section*{Introduction}\label{introduction}}
\addcontentsline{toc}{section}{Introduction}

Parental mentalization refers to parents' ability to understand their children's states of mind and employ this comprehension to interpret and predict their children's behavior (\protect\hyperlink{ref-Fonagy1998b}{Fonagy \& Target, 1998}; \protect\hyperlink{ref-Fonagy1998a}{Fonagy, Target, Steele, \& Steele, 1998}; \protect\hyperlink{ref-Meins1997}{Meins, 1997}; \protect\hyperlink{ref-Pajulo2018}{Pajulo et al., 2018}; \protect\hyperlink{ref-Sharp2008}{Sharp \& Fonagy, 2008}). \emph{Mind-mindedness} and \emph{parental reflective functioning} (PRF) are the two parental mentalization constructs that have been most actively studied to date, and are the focus of the present study. Mind-mindedness is defined as the parent's proclivity to treat their infant as an individual with a mind of their own (\protect\hyperlink{ref-Meins1997}{Meins, 1997}). The original measure assessed mind-mindedness in terms of the extent to which the caregiver focuses on mental and emotional characteristics when invited to describe their preschooler (\protect\hyperlink{ref-Meins1998}{Meins, Fernyhough, Russell, \& Clark-Carter, 1998}). More recently, an observation-based measure was developed to assess mind-mindedness during infancy. This measure operationalizes mind-mindedness in terms of the caregiver's tendency to comment on the infant's internal states in an appropriate versus non-attuned manner (\protect\hyperlink{ref-Meins2001}{Meins, Fernyhough, Fradley, \& Tuckey, 2001}; \protect\hyperlink{ref-Meins2012}{Meins et al., 2012}). Appropriate mind-related comments indicate caregivers' accurate interpretations of the infant's internal state (e.g., stating the infant is interested in a toy car if they have played with it for a sustained period of time), whereas non-attuned mind-related comments index misinterpretations of the infant's thoughts and feelings (e.g., stating the infant is tired in the absence of any behavioral sign of tiredness). Mind-mindedness is characterized as scoring highly for appropriate mind-related comments and/or making no or few non-attuned mind-related comments.

Research over the last 25 years has shown that mind-mindedness predicts a wide range of positive aspects of children's development (see Aldrich, Chen, \& Alfieri (\protect\hyperlink{ref-Aldrich2021}{2021}); McMahon \& Bernier (\protect\hyperlink{ref-McMahon2017}{2017}) for reviews), including caregiver--child attachment security and children's theory of mind abilities. Research also indicates that mind-mindedness may be a protective factor against non-optimal development under challenging caregiving circumstances due to low socioeconomic status (\protect\hyperlink{ref-Meins2013}{Meins, Centifanti, Fernyhough, \& Fishburn, 2013}; \protect\hyperlink{ref-Meins2019}{Meins, Fernyhough, \& Centifanti, 2019}) or premature birth (\protect\hyperlink{ref-Costantini2017}{Costantini, Coppola, Fasolo, \& Cassibba, 2017}).

PRF refers to a parent's ability to represent their child's thoughts, feelings, and beliefs, and thus hold their child's mental experiences in mind (\protect\hyperlink{ref-Slade2005}{Slade, 2005}). It is often assessed from analyzing transcripts of interviews such as the Parent Development Interview (\protect\hyperlink{ref-Aber1985}{Aber, Slade, Berger, Bresgi, \& Kaplan, 1985}), which deal with parenting and the parent--child relationship. A questionnaire measure of PRF has also been developed. The Parental Reflective Functioning Questionnaire (PRFQ; Luyten, Mayes, Nijssens, \& Fonagy (\protect\hyperlink{ref-Luyten2017}{2017})) operationalizes PRF according to three subscales: (a) \emph{Pre-mentalizing modes} indicates parents' tendency not to engage with their children's mental states or to make bizarre mental state attributions, (b) \emph{Certainty about mental states} indexes parents' understanding of the opacity of mental states, and (c) \emph{Interest and curiosity in mental states} assesses parents' genuine interest in their children's thoughts and feelings. Optimal PRF is characterized in terms of lower pre-mentalizing modes, higher interest and curiosity in mental states, and mid-range certainty about mental states (\protect\hyperlink{ref-Luyten2017}{Luyten et al., 2017}).

\hypertarget{cultural-variation-in-mind-mindedness-and-prf}{%
\subsection*{Cultural Variation in Mind-Mindedness and PRF}\label{cultural-variation-in-mind-mindedness-and-prf}}
\addcontentsline{toc}{subsection}{Cultural Variation in Mind-Mindedness and PRF}

Although mind-mindedness has become an established a predictor of children's development, McMahon and Bernier's (\protect\hyperlink{ref-McMahon2017}{2017}) review highlighted the fact that the vast majority of research on mind-mindedness has been conducted on White caregivers, limiting the generalizability of the findings to other ethnic and cultural groups. In the five years since this review was published, three studies have explored cross-cultural differences in mind-mindedness. Hughes, Devine, \& Wang (\protect\hyperlink{ref-Hughes2018}{2018}) reported that Hong-Kong Chinese mothers were less likely than their British counterparts to describe their 4-year-olds in a mind-minded way. Fujita \& Hughes (\protect\hyperlink{ref-Fujita2021}{2021}) reported that, compared with British mothers, Japanese mothers were less mind-minded in their descriptions of their young children, and referred more to their own expectations rather than their children's characteristics.

Dai, McMahon, \& Lim (\protect\hyperlink{ref-Dai2019b}{2019}) conducted the only cross-cultural study using the observational measure of mind-mindedness during infancy. They reported that mainland Chinese mothers made fewer mind-related comments compared to Australian mothers, with Chinese mothers scoring less highly for appropriate comments and more highly for non-attuned comments than their Australian counterparts. Moreover, whereas Australian mothers used ``want'' and ``like'' for referring to the child's desires and preferences significantly more than their Chinese counterparts, Chinese mothers more frequently used ``want'' in non-attuned comments that resulted from their attempts to redirect their infants away from activities in which they were already engaged.

These findings seem to reflect cultural differences in parental values and children's early socialization. As individuality and autonomy are typically emphasized in British and Australian culture, British mothers may be more likely than their Chinese and Japanese counterparts to focus on individual mental states when describing their children. Given that Chinese parenting values emphasize ``training'' (i.e., \emph{guanjiao}) to structure their children's behavior according to collectivistic values (e.g., \protect\hyperlink{ref-Chao1994}{Chao, 1994}), Chinese mothers may be less likely than their Western counterparts to describe their children with reference to their mental characteristics, and more likely to intervene in their infants' activities and state that they want to engage in activities that the parent is suggesting (\protect\hyperlink{ref-Dai2019b}{Dai, McMahon, et al., 2019}). Typical Japanese parenting is based on the concept of \emph{mimamoru} (watching over from a distance) (\protect\hyperlink{ref-Holloway2017}{Holloway, 2017}), which may explain why these parents tended to describe their children in terms of their own expectations rather than their children's characteristics (\protect\hyperlink{ref-Fujita2021}{Fujita \& Hughes, 2021}). Lillard (\protect\hyperlink{ref-Lillard1998}{1998}) argued that while people's attempts to read others' minds may be universal, differences exist across cultures with respect to the specific internal states typically ascribed, as the concept of mind and the value placed on individual internal states vary across cultures.

The main aim of the present study was to investigate cultural differences in mind-mindedness between the United Kingdom (UK) and South Korea. Korean parenting style differs not only from that typically seen in individualistic cultures, but also from the Chinese and Japanese parenting practices discussed above. Korean parenting style has been characterized as a distinct mixture of authoritarian and authoritative styles (\protect\hyperlink{ref-Choi2013}{Y. Choi, Kim, Kim, \& Park, 2013}). Traditional Korean values such as parental virtue (\emph{mo-bum}) are positively associated with parental warmth and acceptance, while the high parental control and corporal punishment typical of Korean parenting are associated with rejection (\protect\hyperlink{ref-Choi2013}{Y. Choi et al., 2013}). Another unique aspect of Korean parenting is its focus on \emph{oneness}---viewing the mother--infant relationship in terms of emotional interdependency, rather than representing the mother and infant as separate individuals (\protect\hyperlink{ref-Jin2012}{Jin, Jacobvitz, Hazen, \& Jung, 2012}; \protect\hyperlink{ref-Kim1994}{U. Kim \& Choi, 1994}; \protect\hyperlink{ref-Kim2005}{Uichol Kim, Park, Kwon, \& Koo, 2005}). Pursuing oneness is implicitly and explicitly manifested in Korean parenting customs; indeed, once a Korean woman becomes a mother, she is referred to as ``{[}child's name's{]} mother'' instead of her first name in social situations involving children. Due to these unique aspects of Korean parenting, we expected levels of parental mentalization in Korean mothers to be different from those observed in their Chinese and Japanese counterparts.

How might traditional Korean parenting practices relate to mind-mindedness? We hypothesized that Korean mothers would produce more appropriate mind-related comments compared with British mothers because the Korean parenting context emphasizes interconnectedness between parents and children. However, as this cultural expectation for mothers and infants to be `at one' may escalate the risk of mothers' projection of their own thoughts and feelings onto their infants, it may also result in Korean mothers being more likely than their British counterparts to make non-attuned mind-related comments. Previous research shows that appropriate and non-attuned mind-related comments are unrelated (e.g., \protect\hyperlink{ref-Meins2012}{Meins et al., 2012}, \protect\hyperlink{ref-Meins2002}{2002}) and should not be considered to represent opposite poles of a unidimesional construct. Some caregivers thus score highly for both appropriate and non-attuned mind-related comments (\protect\hyperlink{ref-Meins2012}{Meins et al., 2012}), and this pattern may be more typical in Korean mothers, given the cultural emphasis on oneness. Moreover, we expected cultural differences to emerge in the types of internal states that mothers attributed to their infants. As Korean mothers' beliefs and goals focus on building relational closeness and achieving oneness with their children (\protect\hyperlink{ref-Park2006}{Park \& Kim, 2006}), a sense of knowing their infants' thoughts and feelings might be more crucial for Korean mothers than for British mothers. Therefore, we anticipated that Korean mothers may concentrate more than their British counterparts on commenting on their infants' emotional and cognitive states. On the other hand, the fact that British mothers rear their infants in an individualistic cultural context that emphasizes independence led us to hypothesize that comments relating to the infant's individual preferences and desires will be more common in British mothers.

The present study also investigated cross-cultural differences in PRF. Lee, Meins, and Larkin (\protect\hyperlink{ref-Lee2021}{2021}) recently developed a Korean translation of the PRFQ. They reported that statements indicative of pre-mentalizing modes did not load onto a single factor for Korean parents, suggesting that this aspect of PRF may not generalize across cultures. They also found that parents' certainty about their children's mental states was positively correlated with self-reported quality of parenting, and proposed that high certainty about mental states may be normative---and deemed optimal---in Korean parents. The construct of oneness characterizes ideal parenting, for mothers in particular, as having privileged and highly accurate insight into the infant's internal states and knowing exactly what the infant is thinking and feeling. The statements in the PRFQ that are indicative of certainty about mental states (e.g., `I always know what my child wants', `I can completely read my child's mind') may thus be seen as socially desirable for Korean mothers. The study reported here was the first to investigate cross-cultural differences in PRF, assessing PRF using the PRFQ rather than an interview in order to explore Lee et al.'s proposal regarding the optimal nature of high certainty about their infants' mental states in Korean mothers. We hypothesized that Korean mothers would achieve higher scores for certainty about mental states compared to their British counterparts. Cultural differences in the other aspects of PRF assessed by the PRFQ (pre-mentalizing modes and interest and curiosity in mental states) were investigated without specific hypotheses. The present study also included both observational and self-report measures of parenting quality to test whether high reported certainty about the infant's mental states was related to more sensitive parenting, and whether any such relation was specific to Korean mothers.

\hypertarget{relations-between-mind-mindedness-and-prf}{%
\subsection*{Relations between Mind-Mindedness and PRF}\label{relations-between-mind-mindedness-and-prf}}
\addcontentsline{toc}{subsection}{Relations between Mind-Mindedness and PRF}

In exploring how culture related to these two different measures of parental mentalization, we were also able to investigate an important but neglected question: how do mind-mindedness and PRF relate to one another? While mind-mindedness and PRF are grouped together under the umbrella term parental mentalization, there is very little research investigating whether the two constructs are related. Establishing the nature of the relation between mind-mindedness and PRF is important both theoretically and empirically. If the two constructs are not robustly related, then this would suggest that parental mentalization is a multidimensional, rather than unidimensional, construct. If this is the case, developmental outcomes associated with mind-mindedness cannot be assumed to generalize to PRF; similarly, PRF may predict aspects of children's development that are not associated with mind-mindedness. Understanding how mind-mindedness and PRF relate to one another is thus critical to mapping the precise developmental trajectories associated with parental mentalization.

Mind-mindedness and PRF are both characterized in terms of the caregiver's engagement with the child's internal states, and may therefore be expected to be positively associated. However, as discussed above, they are operationalized very differently. PRF is indexed purely in terms of the mothers' representations of the child as assessed through interviews or the PRFQ, whereas mind-mindedness is measured from the extent to which the caregiver comments on the infant's internal states in appropriate versus non-attuned ways during actual caregiver--infant interaction. One crucial difference between the constructs is thus that mind-mindedness enables one to judge the accuracy of caregivers' interpretations of their infants' internal states in light of infant behavior, whereas PRF indexes caregivers' more general tendency to engage with and reflect on the child's mental world. PRF may thus be necessary but not sufficient for the caregiver to be mind-minded. In line with this suggestion, mind-mindedness has been reported to be unrelated to parents' underlying theory of mind capacities (\protect\hyperlink{ref-Barreto2016}{Barreto, Fearon, Osório, Meins, \& Martins, 2016}; \protect\hyperlink{ref-Devine2019}{Devine \& Hughes, 2019}).

Two recent studies have investigated relations between mind-mindedness and PRF. Dollberg (\protect\hyperlink{ref-Dollberg2022}{2022}) assessed mind-mindedness and PRF (measured using the Parent Development Interview) concurrently when infants were age 3 months. No associations between PRF and either appropriate or non-attuned comments were found. Krink \& Ramsauer (\protect\hyperlink{ref-Krink2021}{2021}) investigated relations between mind-mindedness and PRF as assessed by the PRFQ in a sample of mothers with postpartum depression. Once again, no associations between mind-mindedness and PRF were observed. In both studies, the effect sizes ranged from negligible to small. The present study was the first to investigate how mind-mindedness related to the PRFQ subscales in a non-clinical sample of mothers and infants. On the basis of previous research, we expected weak associations between mind-mindedness and PRF.

\hypertarget{potential-confounding-factors}{%
\subsection*{Potential Confounding Factors}\label{potential-confounding-factors}}
\addcontentsline{toc}{subsection}{Potential Confounding Factors}

In exploring cultural differences in parental mentalization and relations between mind-mindedness and PRF, it was important to consider a number of potential confounding variables. Asian mothers are known to have generally poorer psychological health compared with Western mothers. For example, comparing across Korean, Israeli, and American women, O'Brien, Ganginis Del Pino, Yoo, Cinamon, \& Han (\protect\hyperlink{ref-OBrien2014}{2014}) found that Korean women reported the highest levels of depression and the least support from their partners. Furthermore, an epidemiological survey of mental disorders in Korea found that diagnosed depression was twice as high in Korean married women than in Korean married men (\protect\hyperlink{ref-Hong2016}{Hong, Lee, \& Ham, 2016}). The disproportionate depression rate in Korean married women has been considered to be partially related to the patriarchal social structure of Korean society---which places restrictions on Korean women's careers in order to emphasize the importance of home-making (\protect\hyperlink{ref-Choi2004}{M. Choi \& Harwood, 2004})---and entangled relationships with mothers-in-law (e.g., \protect\hyperlink{ref-Jones2013}{Jones \& Coast, 2013}). Mothers' psychological wellbeing has also been reported to relate to parental mentalization. Mothers hospitalized for severe mental illness were found to have lower levels of mind-mindedness than their psychologically well counterparts (\protect\hyperlink{ref-Schacht2017}{Schacht et al., 2017}), and PRF is related negatively to psychological distress in both non-clinical (\protect\hyperlink{ref-Ahrnberg2020}{Ahrnberg et al., 2020}; \protect\hyperlink{ref-Rostad2016}{Rostad \& Whitaker, 2016}) and clinical (\protect\hyperlink{ref-Steele2020}{Steele, Townsend, \& Grenyer, 2020}) samples. Similarly, parenting stress has been reported to relate to lower levels of both mind-mindedness (\protect\hyperlink{ref-Dai2019a}{Dai, Lim, \& Xu, 2019}; \protect\hyperlink{ref-Larkin2021}{Larkin et al., 2021}; \protect\hyperlink{ref-McMahon2012}{McMahon \& Meins, 2012}), and PRF (\protect\hyperlink{ref-Nijssens2018}{Nijssens, Bleys, Casalin, Vliegen, \& Luyten, 2018}; \protect\hyperlink{ref-Rutherford2013}{Rutherford, Goldberg, Luyten, Bridgett, \& Mayes, 2013}). We therefore included measures of mothers' psychological wellbeing and parenting stress.

Finally, infant temperament was included as an additional control variable. Krassner et al. (\protect\hyperlink{ref-Krassner2017}{2017}) reported that Korean toddlers were perceived by their mothers to have higher levels of negative affect compared to their Polish and American counterparts. Perceived infant temperament has also been reported to relate to all subscales of the PRFQ. Álvarez, Lázaro, Gordo, Elejalde, \& Pampliega (\protect\hyperlink{ref-Alvarez2022}{2022}) found positive correlations between reported emotional regulation and both interest and curiosity about mental states and certainty about mental states. In contrast, pre-mentalizing modes scores were negatively related to reported emotional regulation. Cultural differences in parental mentalization were therefore explored in light of differences in infant temperament.

In summary, the present study sought to investigate differences between British and Korean mothers in their overall mind-mindedness and in the specific types of internal state on which they commented. Due to the Korean concept of \emph{oneness} in relation to early mother--infant relationships, we hypothesized that (a) Korean mothers would make more appropriate and more non-attuned comments compared with their British counterparts, (b) British mothers would make more comments on their infants' desires and preferences compared with Korean mothers, and (c) Korean mothers would make more comments on their infants' cognitions and emotions compared with British mothers. We also hypothesized that Korean parenting culture would lead to Korean mothers being more likely than British mothers to report high levels of certainty about their infants' mental states. The analyses testing these hypotheses were confirmatory. Exploratory analyses investigated cultural differences in the other two PRFQ subscales (pre-mentalizing modes and interest and curiosity in mental states). Confirmatory analyses investigated whether high certainty about infant mental states was related to more positive parenting, and whether any such relation was specific to Korean mothers. Finally, we explored relations between mind-mindedness and PRF in both nationalities, and expected negligible to small associations between the two measures of parental mentalization. These analyses were also confirmatory.

\hypertarget{method}{%
\section*{Method}\label{method}}
\addcontentsline{toc}{section}{Method}

\hypertarget{participants}{%
\subsection*{Participants}\label{participants}}
\addcontentsline{toc}{subsection}{Participants}

Participants were British mothers (\emph{n} = 63, \emph{M} = 32.51 years, \emph{SD} = 5.97 years, range 22-48 years)
and their infants (37 boys, 26 girls, \emph{M} = 6.14 months, \emph{SD} = 1.55, range 3.5-9.4 months),
and South Korean mothers (\emph{n} = 66, \emph{M} = 33.11 years, \emph{SD} = 3.41 years, range 26--41 years)
and their infants (34 boys, 32 girls, \emph{M} = 7.49 months, \emph{SD} = 1.15, range 4.23--10.63 months).
In the British sample, 61 mothers were White, 1 was Mixed race, and 1 was Asian British;
all 66 mothers in the Korean sample were Asian.
Regarding maternal education, most mothers had at least an undergraduate degree (British mothers: 92.1\%, Korean mothers: 86.4\%).
100\% of Korean mothers and 96.8\% of British mothers were married or in a relationship with the infant's father;
2 British mothers reported that they were not in a relationship.
Both British and Korean mothers were recruited in local communities, through the Internet, and via word of mouth.

\hypertarget{materials-and-methods}{%
\subsection*{Materials and Methods}\label{materials-and-methods}}
\addcontentsline{toc}{subsection}{Materials and Methods}

Testing was conducted in a developmental research laboratory in a session that lasted approximately 20 to 30 minutes. All participants were informed that they provided information anonymously using only ID numbers, and that they could withdraw from the study at any point. Questionnaire measures were completed in an online format. The procedure was approved by the relevant University ethics committees in both the UK and Korea.

\textbf{Maternal Mind-mindedness}. Maternal mind-mindedness was measured via a 10-minute mother--infant free play. The mother was instructed to play with her infant for 10 minutes as she would do at home when they had free time. They were situated in a room set up with a range of age-appropriate toys and a video recorder. Mothers' speech was transcribed verbatim and was later coded according to the mind-mindedness coding manual (\protect\hyperlink{ref-Meins2015}{Meins \& Fernyhough, 2015}). Mind-related comments include mothers' comments about the infants' (a) desires and preferences, (b) cognitions, (c) emotions, (d) epistemic states, and (e) talking on the infant's behalf. Each mind-related comment was then classed as appropriate or non-attuned by a trained coder. Mind-related comments were classified as appropriate if any of the following criteria were met: (a) the coder agreed with the mother's reading of the infant's current internal state, (b) the mother's comment linked the infant's current internal state with similar events in the past or future, or (c) the comment suggested the infant would like or want a new object or activity after a lull in the interaction. Comments were classified as non-attuned if: (a) the coder disagreed with the mother's reading of the infant's current internal state, (b) the mother's comment regarding a past or future event was not related to the infant's current internal state, (c) the mother asked what the infant wanted to do or suggested a new activity when the infant was already involved in something else, (d) the comment seemed not to be based on the infant's behavior or seemed to project the mother's internal states onto the infant, or (e) the referent of the comment was not clear. To control for verbosity, appropriate and non-attuned mind-related comments were calculated as a proportion of the total number of comments made during the play session.

For coding mind-mindedness in the Korean sample, the original English manual was translated into Korean by a bilingual researcher, and the validity of the translated coding system was carefully considered with two other bilingual Korean/English speakers, one of whom was a developmental clinical psychologist and the other a developmental practitioner. As the main coder of the Korean sample (the first author) was the same as the main coder of the British sample, the consistency of coding between the two cultural samples was maintained. A randomly selected 20\% of the free-play sessions for each cultural group was coded by a second trained coder, who was a native speaker of English or Korean, and who was blind to all other measures. Inter-rater reliability was UK \(\kappa\) = .72, and KOR \(\kappa\) = .86. Disagreements were resolved by discussion.

In addition, the mothers' mind-related comments were classified into categories regarding the content of the comments to further investigate differences in mothers' mind-related comments across cultures. The mind-related comments were coded as one of the following exhaustive and exclusive categories described in Meins and Fernyhough's (\protect\hyperlink{ref-Meins2015}{2015}) coding manual: (a) desire and preference (e.g., ``Do you like the ball?''), (b) cognitions (e.g., ``Do you know what it is?''), (c) intention (e.g., ``Are you trying to put that one in?''), (d) emotions (e.g., ``Are you happy?''), (e) epistemic states (e.g., ``Are you playing games with me?''), (f) talking on the infant's behalf (any utterance that is obviously meant to be said or thought by the infant, e.g., ``I don't like it, mum''), (g) physical statement (e.g., ``Are you tired?''). Note that terms in the final category can only be non-attuned if they are categorised as mind-related (Meins \& Fernyhough, 2015). There was perfect agreement between first and second trained coders with regard to the mental state categories.

\textbf{Parenting Reflective Functioning}. The Parental Reflective Functioning Questionnaire (PRFQ; Luyten et al. (\protect\hyperlink{ref-Luyten2017}{2017})) was completed to assess parents' PRF. The PRFQ is an 18-item questionnaire using a 7-point Likert scale (1: strongly disagree to 7: strongly agree), designed to assess multidimensional traits of PRF with three subscales: (a) \emph{Pre-mentalizing modes} (6 items), (b) \emph{Certainty about mental states} (6 items), and (c) \emph{Interest and curiosity in mental states} (6 items). The subscale of \emph{Pre-mentalizing modes} measures parents' inability to enter their children's mental states and their tendency to make malevolent attributions (e.g., ``When my child is fussy, he or she does that just to annoy me''). A high score on this subscale indicates parents' non-mentalizing stance. The \emph{Certainty about mental states} subscale captures parents' tendency to be highly certain about their children's mental states (e.g., ``I always know what my child wants''). A high score on this subscale reflects lack of awareness of the opacity of mental states, while an extremely low score reflects difficulty in having confidence about the child's subjective world. Lastly, the \emph{Interest and curiosity in mental states} subscale measures parents' genuine interest and curiosity in their children's subjective world (e.g., ``I like to think about the reasons behind the way my child behaves and feels''). A high score on this subscale indicates parents' high interest in their children's mental states. In total, the PRFQ gives each mother three separate average scores for the three subscales, ranging from 1 to 7. The British mothers completed the PRFQ in English; the Korean mothers completed the Korean translation of the PRFQ (\protect\hyperlink{ref-Lee2021}{Y. Lee et al., 2021}).

Acceptable internal reliability has been reported for the three scales in the English version of the PRFQ (\protect\hyperlink{ref-Anis2020}{Anis et al., 2020}), but a recent study reported low internal reliability for the pre-mentalizing subscale (\protect\hyperlink{ref-Arkle2023}{Arkle et al., 2023}). In the Korean version of the PRFQ, Y. Lee et al. (\protect\hyperlink{ref-Lee2021}{2021}) reported acceptable internal reliability for the certainty about mental states and curiosity about mental states subscales, but low internal reliability for the pre-mentalizing subscale. Internal reliability for the present study was as follows: \emph{Pre-mentalizing modes}, \(\alpha_{UK}\) = .72, and \(\alpha_{KOR}\) = .40; \emph{Certainty about mental states}, \(\alpha_{UK}\) = .64, and \(\alpha_{KOR}\) =.83; \emph{Interest and curiosity in mental states}, \(\alpha_{UK}\) = .76, and \(\alpha_{KOR}\) =.71. Due to the low reliability of the Korean \emph{Pre-mentalizing modes} subscale, results for this subscale were not included in the statistical analyses.

\textbf{Parenting Style}. The Revised Parents as Social Context Questionnaire (R-PSCQ; Egeli, Rogers, Rinaldi, \& Cui (\protect\hyperlink{ref-Egeli2015}{2015}); Skinner, Johnson, \& Snyder (\protect\hyperlink{ref-Skinner2005}{2005})) was used to assess multiple aspects of parenting style. It consists of 30 items rated on a 4-point Likert scale (1: not at all true to 4: very true) to index the following dimensions of parenting: (a) warmth (e.g., ``I set aside time to talk to my child about what is important to him/her''); (b) rejection (e.g., ``Sometimes, my child is hard to like''); (c) structure (e.g., ``I expect my child to follow our family rules''); (d) chaos (e.g., ``When my child gets in trouble, my reaction is not very predictable''); (e) autonomy support (e.g., ``I trust my child''); and (f) coercion (e.g., ``My child fights me at every turn''). Each subscale's score ranges between 5 and 20, and a higher score indicates a greater level of the corresponding parenting dimension. Egeli et al. (\protect\hyperlink{ref-Egeli2015}{2015}) reported acceptable internal reliability for all subscales of the R-PSCQ. Korean parents completed the adapted Korean version of the PSCQ (\protect\hyperlink{ref-Jeong2011}{Jeong \& Shin, 2011}). Jeong \& Shin (\protect\hyperlink{ref-Jeong2011}{2011}) reported a good fit between the Korean and the original version factor structures, with acceptable reliability for all subscales. Internal reliability for the present study was as follows: warmth \(\alpha_{UK}\) = .66, and \(\alpha_{KOR}\) = .79; rejection \(\alpha_{UK}\) = .67, and \(\alpha_{KOR}\) = .68; structure \(\alpha_{UK}\) = .72, and \(\alpha_{KOR}\) = .70; chaos \(\alpha_{UK}\) = .78, and \(\alpha_{KOR}\) = .61; autonomy support \(\alpha_{UK}\) = .64, and \(\alpha_{KOR}\) = .60; and coercion \(\alpha_{UK}\) = .75, and \(\alpha_{KOR}\) = .77.

Following Skinner et al. (\protect\hyperlink{ref-Skinner2005}{2005}), composite scores were calculated to represent positive parenting style (the sum of scores for the warm, structured, and autonomy-supportive subscales), and negative parenting style (the sum of scores for the rejective, chaotic, and coercive subscales). The positive and negative parenting style composites had acceptable internal reliability, \(\alpha\) = .64 and .66, respectively.

\textbf{Maternal Intrusiveness}. The free-play sessions from which mind-mindedness was coded were also used to assess maternal intrusiveness using Miller and Sameroff's (\protect\hyperlink{ref-Miller1998}{1998}) coding manual. Given that the original coding system was for 3-minute interactions, we divided the 10-minute session into three 3-minute 20-second epochs. The total score was averaged over the three epochs. The level of intrusiveness in each epoch was coded by the degree to which a mother handled her infant roughly on scales of 0 (no intrusiveness) to 3 (predominant or high intrusiveness). A randomly selected 20\% of the free-play sessions was double coded by a second trained coder, who was blind to other measures. Inter-rater reliability was ICC\textsubscript{UK} = .92, and ICC\textsubscript{KOR} = .92.

\textbf{Maternal Anxiety and Depression}. The Hospital Anxiety and Depression Scale (HADS; Zigmond \& Snaith (\protect\hyperlink{ref-Zigmond1983}{1983})) was used to measure parental mental health. The HADS is a 14-item questionnaire using a 4-point Likert scale ranging from 0 to 3. It has two subscales: anxiety (HADS-A), and depressive symptoms (HADS-D). Each of the two subscales has a range between 0 and 21, and the total score range is 0 to 42. A higher score indicates a greater level of anxiety/depressive symptoms. The HADS is a highly reliable and valid measure of mental health (\protect\hyperlink{ref-Bjelland2002}{Bjelland, Dahl, Haug, \& Neckelmann, 2002}). The Korean Hospital Anxiety and Depression Scale (K-HADS; Oh, Min, \& Park (\protect\hyperlink{ref-Oh1999}{1999})) was used for Korean parents. Oh et al. (\protect\hyperlink{ref-Oh1999}{1999}) reported acceptable internal reliability for all subscales of the Korean version of the R-PSCQ. Composite scores for the two subscales were used in the analyses. Internal reliability was as follows: \(\alpha_{UK}\) = .82, and \(\alpha_{KOR}\) = .87.

\textbf{Parenting Stress}. The parents completed the Parenting Stress Index-Short Form (PSI-SF; Abidin (\protect\hyperlink{ref-Abidin1990}{1990})), which is a 36-item questionnaire using a 5-point Likert scale (1: strongly agree to 5: strongly disagree). The questionnaire consists of three subscales: parental distress, parent--child dysfunctional interaction, and difficult child. The parental distress scale reflects a parent's perceived distress related to parenting (e.g., conflict with spouse, social support, and restrictions caused by having a child), the parent--child dysfunctional interaction scale reflects a parent's perception that interactions with the child do not meet expectations, and the difficult child scale reflects the parent's perception of the child's characteristics and the extent to which dealing with the child is difficult (e.g., demandingness, temper tantrums). Each subscale's score ranges from 12 to 60, and the total score across the three subscales indicates the overall level of parenting stress, ranging from 36 to 180. Abidin (\protect\hyperlink{ref-Abidin1995}{1995}) reported good internal reliability for all subscales of the PSI-SF. For Korean mothers, the Korean Parenting Stress Index-Short Form (K-PSI-SF; K. S. Lee, Chung, Park, \& Kim (\protect\hyperlink{ref-Lee2008b}{2008})) was used. K. S. Lee et al. (\protect\hyperlink{ref-Lee2008b}{2008}) reported acceptable internal reliability for all subscales of the Korean version of the PSI-SF. Due to the different direction of scales of the K-PSI-SF (1: strongly disagree to 5: strongly agree) from the English version, the scores of the English PSI-SF were reversed, and thus a low raw score indicated a low level of stress related to parenting. Internal reliability was as follow: parenting distress, \(\alpha_{UK}\) = .84, and \(\alpha_{KOR}\) = .81; parent--child dysfunctional interaction, \(\alpha_{UK}\) = .88, and \(\alpha_{KOR}\) = .83; difficult child, \(\alpha_{UK}\) = .87, and \(\alpha_{KOR}\) = .88.

Scores for the parenting distress and parent--child dysfunctional interaction scales were used as measures of parenting stress, and scores for the difficult child scale were used to index perceived infant temperament.

\textbf{Observed Infant Temperament}. The car seat task from the Infant Laboratory Temperament Assessment Battery (Lab-TAB; Goldsmith \& Rothbart (\protect\hyperlink{ref-Goldsmith1996}{1996})) was used for measuring observed infant temperament. It was designed to elicit mild anger responses from infants by restraining them for 30 seconds in a car seat, secured to a chair in the laboratory; a video camera was positioned to record the infant's face and body movements. This task has good predictive and concurrent validity for assessing infant temperament when used in isolation from the whole Lab-TAB battery (\protect\hyperlink{ref-Hay2014}{Hay et al., 2014}; \protect\hyperlink{ref-Larkin2019}{Larkin, Oostenbroek, Lee, Hayward, \& Meins, 2019}). In order to score the episode, the 30 seconds were divided into six 5-second epochs to code intensity of (a) facial anger (0: no facial expression to 3: a change occurs in all 3 facial regions or gives an impression of strong anger), (b) facial sadness (0: no facial expression to 3: a change occurs in all 3 facial regions or gives an impression of strong sadness), (c) distress vocalizations (0: no distress to 5: full intensity cry/scream), and (d) struggle (0: no struggle at all to 4: high intensity struggle). Average scores for each category were calculated across the epochs. Higher scores indicate greater negative affect, distress, and struggle, suggestive of a more difficult temperament. A randomly selected 20\% of the Lab-TAB sessions was double coded by a second trained coder, who was blind to other measures. Inter-rater reliability for the ratings were: facial anger, ICC\textsubscript{UK} = .97, and ICC\textsubscript{KOR} = .77; facial sadness, ICC\textsubscript{UK} = .89, and ICC\textsubscript{KOR} = .93; distress vocalization, ICC\textsubscript{UK} = .97, and ICC\textsubscript{KOR} = .94; struggle, ICC\textsubscript{UK} = .80, and ICC\textsubscript{KOR} = .68. There was good reliability for a composite measure of the four ratings, \(\alpha_{UK}\) = .81, and \(\alpha_{KOR}\) = .86. The composite measure was used in the analyses.

\hypertarget{data-availability}{%
\subsection*{Data Availability}\label{data-availability}}
\addcontentsline{toc}{subsection}{Data Availability}

The data file is available via the Open Science Framework (\url{https://osf.io/mpn95/?view_only=7170614a3fed4f9a8b0498223fb3a230}). The study was not preregistered.

\hypertarget{results}{%
\section*{Results}\label{results}}
\addcontentsline{toc}{section}{Results}

\hypertarget{descriptive-statistics-and-preliminary-analyses}{%
\subsection*{Descriptive Statistics and Preliminary Analyses}\label{descriptive-statistics-and-preliminary-analyses}}
\addcontentsline{toc}{subsection}{Descriptive Statistics and Preliminary Analyses}

Two British mothers did not answer the age question, and another British mother's HADS and PSI-SF responses could not be used due to a technical problem. The mother with missing HADS and PSI-SF data was excluded from the main analyses on cultural differences in parental mentalization reported below.
Korean infants were older than British infants, \emph{t}(127) = 5.63, \emph{p} = \textless0.001,
and British mothers were more highly educated than Korean mothers, \emph{t}(127) = 3.05, \emph{p} = 0.003,
but the groups did not differ on maternal age, \emph{t}(125) = 0.69, \emph{p} = 0.493.
Infant age and maternal education were therefore included as additional covariates in the cultural differences analyses reported below. With regard to the separate categories of mind-related comments, only two mothers mentioned epistemic states and only eight mothers talked on the infant's behalf. These categories are therefore not used in the analyses below. Table 1 shows the parenting style, mental health, and infant temperament data as a function of nationality. As shown Table 1, British mothers showed more positive parenting than did their Korean counterparts for both the self-report and observational measures. In terms of psychological well-being, Korean mothers scored more highly than British mothers for anxiety/depression and parenting distress. However, there were no significant differences between British and Korean mothers for reported or observed infant temperament and parent--child dysfunctional interaction (see Table 1).

\hypertarget{cultural-differences-in-parental-mentalization}{%
\subsection*{Cultural Differences in Parental Mentalization}\label{cultural-differences-in-parental-mentalization}}
\addcontentsline{toc}{subsection}{Cultural Differences in Parental Mentalization}

Table 2 shows the mind-mindedness data for the UK and Korean samples. Cultural differences in appropriate and non-attuned mind-related comments were investigated using ANCOVA, with nationality added as a fixed variable and infant age, maternal education, HADS scores, the PSI-SF subscale scores (parenting distress, parent--child dysfunctional interaction, difficult child), and Lab-TAB scores added as covariates. The correlations between the variables are shown in supplemental material 1.
There was no main effect of nationality for appropriate mind-related comments, \emph{F}(8, 119) = 0.07, \emph{p} = 0.790,
or for non-attuned mind-related comments, \emph{F}(8, 119) = 0.18, \emph{p} = 0.668.

Table 2 also shows the data for the separate categories of mind-related comments. Cultural differences in the separate categories of appropriate mind-related comments were investigated using MANCOVA, with the separate categories added as dependent variables, nationality added as a fixed variable, and infant age, maternal education, HADS scores, the PSI-SF subscale scores (parenting distress, parent--child dysfunctional interaction, difficult child), and Lab-TAB scores added as covariates.
There was a main effect of nationality, \emph{F}(4, 116) = 3.93, \emph{p} = 0.005.
Follow-up tests showed a main effect of nationality on appropriate comments on (a) desires and preferences, \emph{F}(1, 119) = 11.41, \emph{p} = \textless0.001,
(b) cognition, \emph{F}(1, 119) = 4.95, \emph{p} = 0.028,
and (c) emotion, \emph{F}(1, 119) = 8.38, \emph{p} = 0.005;
there was no main effect of nationality on intention, \emph{F}(119, 1) = 0.52, \emph{p} = 0.473.
British mothers commented more frequently in appropriate ways on their infants' desires and preferences than did Korean mothers, but Korean mothers made more appropriate mind-related comments about their infants' cognitive and emotional states compared with British mothers (see Table 2).

Cultural differences in the separate categories of non-attuned mind-related comments were investigated using MANCOVA as described above.
There was a main effect of nationality, \emph{F}(5, 115) = 4.75, \emph{p} = \textless0.001.
Follow-up tests showed a main effect of nationality on non-attuned comments on (a)
desires and preferences, \emph{F}(1, 119) = 6.1, \emph{p} = 0.015, (b)
emotion, \emph{F}(1, 119) = 4.39, \emph{p} = 0.038, and (c)
physical states, \emph{F}(1, 119) = 12.78, \emph{p} = \textless0.001;
there was no main effect of nationality on cognition, \emph{F}(1, 119) = 2.47, \emph{p} = 0.119,
or intention, \emph{F}(119, 1) = 0.99, \emph{p} = 0.323.
British mothers commented more frequently in non-attuned ways on their infants' desires and preferences than did Korean mothers, but Korean mothers made more non-attuned mind-related comments about their infants' emotional and physical states compared with British mothers (see Table 2).

The PRF data are shown in Table 2 as a function of nationality. Cultural differences in PRF were investigated using MANCOVA, with scores for \emph{Certainty about mental states} and \emph{Interest and curiosity about mental states} added as dependent variables, nationality added as a fixed variable, and infant age, maternal education, HADS scores, the PSI-SF subscale scores (parenting distress, parent--child dysfunctional interaction, difficult child), and Lab-TAB scores added as covariates. The correlations between the variables are shown in supplemental material 1. There was a main effect of nationality, \emph{F}(2, 118) = 15.1, \emph{p} = \textless0.001.
Follow-up tests showed a main effect of nationality on \emph{Certainty about mental states},
\emph{F}(1, 119) = 26.7, \emph{p} = \textless0.001, but not \emph{Interest and curiosity in mental states},
\emph{F}(1, 119) = 0.61, \emph{p} = 0.435. As shown in Table 2,
Korean mothers scored more highly than British mothers for \emph{Certainty about mental states}.

\hypertarget{relations-between-parenting-and-certainty-about-infant-mental-states}{%
\subsection*{Relations between Parenting and Certainty about Infant Mental States}\label{relations-between-parenting-and-certainty-about-infant-mental-states}}
\addcontentsline{toc}{subsection}{Relations between Parenting and Certainty about Infant Mental States}

Our final hypothesis related to whether mothers' reported certainty about their infants' mental states was related to more positive parenting, and whether any such relation was specific to Korean mothers. This hypothesis was investigated using self-reported parenting quality and an observational measure of maternal intrusiveness. Self-reported and observed parenting measures were unrelated in both Korean and British mothers.
For Korean mothers, observed parenting was unrelated to reported positive parenting, \emph{r}(64) = 0.02, \emph{p} = 0.858, and reported negative parenting,
\emph{r}(64) = -0.05, \emph{p} = 0.709.
For British mothers, observed parenting was unrelated to reported positive parenting, \emph{r}(61) = -0.05, \emph{p} = 0.700, and reported negative parenting,
\emph{r}(61) = 0.14, \emph{p} = 0.262.
Scores for \emph{Certainty about mental states} were (a) positively correlated with self-reported positive parenting in Korean mothers, \emph{r}(64) = 0.46, \emph{p} = \textless0.001, and British mothers, \emph{r}(61) = 0.4, \emph{p} = 0.001,
and (b) negatively correlated with self-reported negative parenting in Korean mothers, \emph{r}(64) = -0.47, \emph{p} = \textless0.001, and British mothers, \emph{r}(61) = -0.39, \emph{p} = 0.002.
In contrast, observed maternal intrusiveness was not related to \emph{Certainty about mental states} in either Korean mothers, \emph{r}(64) = -0.13, \emph{p} = 0.306, or British mothers, \emph{r}(61) = -0.19, \emph{p} = 0.131.

\hypertarget{relations-between-maternal-mind-mindedness-and-prf}{%
\subsection*{Relations between Maternal Mind-mindedness and PRF}\label{relations-between-maternal-mind-mindedness-and-prf}}
\addcontentsline{toc}{subsection}{Relations between Maternal Mind-mindedness and PRF}

Finally, relations between the two constructs of parental mentalization were explored. Correlations between appropriate and non-attuned mind-related comments and the PRF subscales were run separately for the British and Korean mothers. As shown in Table 3, there were no relations between the mind-mindedness and PRF variables in either British or Korean mothers, with negligible to small effects for all correlations. Given that effect sizes in this range were predicted, additional Bayesian analyses were conducted to verify the null findings (see Table 3). We set the Bayes factor to BF01 to indicate the strength of evidence in favor of the null hypothesis. For the UK sample, the Bayesian analyses indicate strong support for the null hypothesis for the correlation between appropriate mind-related comments and \emph{Certainty about mental states}, and moderate support for the null hypothesis for the correlation between (a) appropriate comments and \emph{Interest and curiosity in mental states}, (b) non-attuned comments and \emph{Certainty about mental states}, and (c) non-attuned comments and \emph{Interest and curiosity in mental states}. For the Korean sample, the Bayesian analyses indicate moderate support for the null hypothesis for the correlation between (a) appropriate mind-related comments and \emph{Certainty about mental states}, (b) appropriate comments and \emph{Interest and curiosity in mental states}, and (c) non-attuned comments and \emph{Certainty about mental states}, and anecdotal support for the null hypothesis for the correlation between non-attuned comments and \emph{Interest and curiosity in mental states}. Note that the correlations between \emph{Interest and curiosity in mental states} and non-attuned mind-related comments are positive in the Korean sample.

\hypertarget{discussion}{%
\section*{Discussion}\label{discussion}}
\addcontentsline{toc}{section}{Discussion}

The present study expands our understanding of cultural differences in parental mentalization. We hypothesized that the focus on \emph{oneness} in Korean parenting would make Korean mothers more likely than their British counterparts to comment in both appropriate and non-attuned ways on their infants' internal states. Contrary to our hypotheses, no differences in either index of mind-mindedness were observed across the two cultures. These findings contrast with those in the extant literature indicating that mothers from other Asian cultures are less mind-minded than their Western counterparts (\protect\hyperlink{ref-Dai2019b}{Dai, McMahon, et al., 2019}; \protect\hyperlink{ref-Fujita2021}{Fujita \& Hughes, 2021}; \protect\hyperlink{ref-Hughes2018}{Hughes et al., 2018}). As discussed in the Introduction, typical Korean parenting represents a unique blend of authoritarian and authoritative styles (\protect\hyperlink{ref-Choi2013}{Y. Choi et al., 2013}), and the lower levels of mind-mindedness observed in Chinese and Japanese mothers compared with their British and Australian counterparts may therefore not provide strong grounds for expecting similar differences between Korean and Western mothers. Our findings thus highlight the need for caution in assuming equivalence in parenting across collectivistic Asian cultures and making a false dichotomy between ``Western mind'' versus ``Eastern mind'' (\protect\hyperlink{ref-Harkness2000}{Harkness, Super, \& Tijen, 2000}).

Our hypothesized cultural differences were, however, observed in relation to the type of internal states on which Korean and British mothers commented. British mothers commented more than their Korean counterparts on their infants' desires and preferences, whereas Korean mothers focused more on their infants' cognitions and emotions. Korean mothers were also more likely than British mothers to make non-attuned comments about their infants' physical states. These findings likely reflect British mothers' parenting goals of facilitating their infants' individuality and agency, in line with their individualistic cultural context. Indeed, during the free play sessions, British mothers often started their play by asking about their infant's preference for different toys (e.g., ``What would you like to play with?''), while most Korean mothers started by asking about an object label (e.g., ``What is this?''). This is consistent with Keller et al.'s (\protect\hyperlink{ref-Keller2007}{2007}) finding that German mothers engaged in more autonomy-supportive verbal discourse during free play compared with urban Chinese mothers. Korean mothers' observed emphasis on cognition, emotion, and physical states compared with their British counterparts appears in line with their parenting goal of establishing relational closeness with their infants, and may thus represent a manifestation of oneness. Together with other findings in the extant literature, our results suggest that maternal mind-mindedness is universal, but that specific cultural contexts result in differences in the emphases placed on certain internal states. This is in line with Lillard's (\protect\hyperlink{ref-Lillard1998}{1998}) proposal that the specific features of mental states that are ascribed---rather than the overall propensity to talk about the mental states of others---differ depending on how the particular culture comprehends the concept of mind and its development.

Interestingly, these differences in the types of internal state emphasized in the two cultural contexts did not necessarily result in greater accuracy in interpreting such internal states. British mothers scored more highly for both appropriate and non-attuned comments on desires and preferences than did Korean mothers, while Korean mothers scored more highly for both appropriate and non-attuned comments relating to their infants' emotions. These findings raise the intriguing possibility that a cultural focus on the importance of certain internal states may result in caregivers overinterpreting those states emphasized within their particular cultural context. For example, British individualistic culture may mean that caregivers interpret subtle cues as indicating that the infant wants a particular toy when in fact the infant may not evidence such a desire. Similarly, the Korean focus on oneness may increase the likelihood that mothers will attribute emotional or physical states to their infants in the absence of any behavioral cue. Although there is an established literature on the developmental outcomes of mind-mindedness, with appropriate mind-related comments predicting positive aspects of development and non-attuned comments predicting less adaptive aspects of development (\protect\hyperlink{ref-Meins2013}{Meins et al., 2013}; e.g., \protect\hyperlink{ref-Meins2012}{Meins et al., 2012}), research has not yet charted how such predictive relations may be moderated by the types of internal state on which caregivers comment. Given the cultural differences with regard to the content of internal state speech observed in the present study, future research should consider differentiating not only between appropriate and non-attuned mind-related comments, but also different categories of internal states. These more fine-grained analyses of the content of caregivers' mind-related comments may prove important in understanding how mind-mindedness predicts children's development across cultures. Future research could explore this issue cross-culturally in relation to established correlates of early mind-mindedness, such as attachment security and children's mentalizing abilities.

Turning to differences in PRF, the results supported the hypothesized higher scores for \emph{Certainty about mental states} in Korean mothers compared to their British counterparts. In contrast, the two groups did not differ with respect to \emph{Interest and curiosity in mental states}. Considering that the Korean parenting context emphasises the sense of oneness between mother and infant, it is not surprising that Korean mothers are more likely than British mothers to consider themselves to know precisely what their infants are thinking and feeling. The \emph{Certainty about mental states} and \emph{Interest and curiosity in mental states} subscales had acceptable internal reliability in both cultural groups, suggesting that these aspects of PRF generalize across British and Korean cultures. However, the \emph{Pre-mentalizing modes} subscale was found to have low internal reliability in the Korean sample of mothers. Y. Lee et al. (\protect\hyperlink{ref-Lee2021}{2021}) reported that statements indicative of pre-mentalizing modes did not load onto a single factor for Korean parents. This aspect of PRF may thus not show cross-cultural generalizability. Interestingly, Arkle et al. (\protect\hyperlink{ref-Arkle2023}{2023}) assessed PRF longitudinally in a British sample and reported low internal reliability in the \emph{Pre-mentalizing modes} subscale at both testing phases. These findings suggest that high levels of variance in this subscale may not apply only to Korean parents. Variability in the internal reliability of \emph{Pre-mentalizing modes} may be due to the fact that this subscale focuses on diverse aspects of parenting, including a range of ways in which caregivers may attribute malicious intent to the child (``My child sometimes gets sick to keep me from doing what I want to do'', ``When my child is fussy he or she does that just to annoy me'', ``My child cries around strangers to embarrass me''), lack of engagement with the child's mental world (``I find it hard to actively participate in make believe play with my child'', ``Often, my child's behavior is too confusing to bother figuring out''), and uncertainty about being loved by the child (``The only time I'm certain my child loves me is when he or she is smiling at me''). Future research should consider developing more culturally appropriate and sensitive measures of this aspect of parental mentalization, and further investigate whether the \emph{Pre-mentalizing modes} subscale comprises separate dimensions of PRF.

The present study was the first to investigate relations between the two parental mentalization constructs using a cross-cultural design. No associations were observed between any of the indices of mind-mindedness and PRF in either cultural group. These findings replicate the null results reported for relations between mind-mindedness and PRF in previous research (\protect\hyperlink{ref-Dollberg2022}{Dollberg, 2022}; \protect\hyperlink{ref-Krink2021}{Krink \& Ramsauer, 2021}), with effect sizes for the present study in line with the negligible to small effects previously reported. We used Bayesian analyses to confirm the findings, with these analyses indicating moderate to strong support for the null hypothesis for all relations apart from that between non-attuned comments and \emph{Interest and curiosity in mental states} in Korean mothers, for which there was only anecdotal support for the null hypothesis. However, this correlation between non-attuned comments and \emph{Interest and curiosity in mental states} was positive, indicating that mothers' tendency to misinterpret their infants' thoughts and feelings during infant--mother interaction was associated with higher self-reported interest and curiosity in their infants' internal states. These findings thus present robust evidence that mind-mindedness and PRF are unrelated.

There is thus a growing body of research indicating that the caregiver's tendency to reflect on the infant's mind in the context of an interview or questionnaire does not mean that caregivers will translate this reflection into their actual interactions with their infants, and consequently demonstrate higher levels of mind-mindedness. These findings are relevant to our understanding of the umbrella construct of parental mentalization. The fact that a number of studies have reported null findings for relations between mind-mindedness and PRF highlights the need for caution in assuming that the two constructs assess the same facet of caregiving, and support conceptualizing parental mentalization as a multidimensional construct. Further null results may necessitate a rethinking of theoretical perspectives on caregivers' more general tendency to engage with their children's mental worlds, and result in a move away from grouping constructs such as mind-mindedness and PRF under the label parental mentalization.

Our final aim was to investigate whether the \emph{Certainty about mental states} aspect of PRF related to the quality of parenting, and whether any such relation was specific to Korean mothers. Higher \emph{Certainty about mental states} was positively correlated with self-reported parenting quality in both Korean and British mothers. These findings are in line with those of Y. Lee et al. (\protect\hyperlink{ref-Lee2021}{2021}), who found that Korean mothers' reported optimal parenting style was positively related to scores on the \emph{Certainty about mental states} subscale. However, our findings indicate that this relation generalizes to British mothers, and thus appears not to result from having precise insight into the infant's internal states being regarded as ideal for Korean mothers. Regardless of the cultural parenting context, mothers who perceive themselves to have a positive parenting style showed higher levels of certainty about what their infants were thinking and feeling. In contrast, certainty about mental states was unrelated to observed parenting quality in both cultural groups. Greater certainty did not therefore translate into more sensitive parenting. The fact that the association with certainty about mental states was seen only for the self-report measure of parenting suggests that shared method variance may underlie the observed association.

Our results should be interpreted in light of certain limitations. First, the \emph{Pre-mentalizing modes} subscale was found to have low reliability in Korean mothers, and scores for this subscale were therefore not included in the analyses. The present study was therefore unable to compare differences in this aspect of PRF cross-culturally, or to investigate how scores on this subscale of PRF related to the mind-mindedness indices. Second, as most of the participants in the present study were from educated, middle-class backgrounds, it cannot be assumed that our findings will generalize to other populations.

In future research, it would be interesting to investigate how these cultural differences in parental mentalization predict children's later development. Certainty about the mental states of one's infant is considered to indicate a lack of awareness of the opacity of mental states, and is thus taken to indicate lower levels of PRF and less optimal parenting in Western cultures. However, the same might not be true of cultures such as Korea. If such certainty is culturally appropriate, it may be associated with positive aspects of child development. While findings like this would appear counter-intuitive when interpreted with reference to Western parenting ideals, the same does not hold from the perspective of Korean parenting ideals. On the other hand, although maternal mind-mindedness is known to predict children's mentalizing ability in Western countries (e.g., \protect\hyperlink{ref-Kirk2015}{Kirk et al., 2015}; \protect\hyperlink{ref-Laranjo2010}{Laranjo, Bernier, Meins, \& Carlson, 2010}; \protect\hyperlink{ref-Lundy2013}{Lundy, 2013}; \protect\hyperlink{ref-Meins2003}{Meins et al., 2003}), little is known about the link between parental mentalization and the development of children's mentalizing skills in non-Western countries. Sensitive future research on different populations and cultures can therefore enrich our understanding of the parenting role and the many ways in which parenting relates to children's development.

\newpage

\hypertarget{references}{%
\section*{References}\label{references}}
\addcontentsline{toc}{section}{References}

\hypertarget{refs}{}
\begin{CSLReferences}{1}{0}
\leavevmode\vadjust pre{\hypertarget{ref-Aber1985}{}}%
Aber, J. L., Slade, A., Berger, B., Bresgi, I., \& Kaplan, M. (1985). \emph{The parent development interview: Interview protocol}. Unpublished manuscript: Barnard College, Columbia University, New York.

\leavevmode\vadjust pre{\hypertarget{ref-Abidin1990}{}}%
Abidin, R. R. (1990). \emph{Parenting stress index - short form}. Charlottesville, VA: Pediatric Psychology Press.

\leavevmode\vadjust pre{\hypertarget{ref-Abidin1995}{}}%
Abidin, R. R. (1995). \emph{Parenting stress index: Professional manual; {[}PSI{]}}. PAR, Psychological Assessment Resources.

\leavevmode\vadjust pre{\hypertarget{ref-Ahrnberg2020}{}}%
Ahrnberg, H., Pajulo, M., Scheinin, N. M., Karlsson, L., Karlsson, H., \& Karukivi, M. (2020). Association between parental alexithymic traits and self-reported postnatal reflective functioning in a birth cohort population. Findings from the FinnBrain birth cohort study. \emph{Psychiatry Res}, \emph{286}, 112869. \url{https://doi.org/10.1016/j.psychres.2020.112869}

\leavevmode\vadjust pre{\hypertarget{ref-Aldrich2021}{}}%
Aldrich, N. J., Chen, J., \& Alfieri, L. (2021). Evaluating associations between parental mind-mindedness and children's developmental capacities through meta-analysis. \emph{Developmental Review}, \emph{60}, 100946. https://doi.org/\url{https://doi.org/10.1016/j.dr.2021.100946}

\leavevmode\vadjust pre{\hypertarget{ref-Alvarez2022}{}}%
Álvarez, N., Lázaro, M. H., Gordo, L., Elejalde, L. I., \& Pampliega, A. M. (2022). Maternal mentalization and child emotion regulation: A comparison of different phases of early childhood. \emph{Infant Behav Dev}, \emph{66}, 101681. \url{https://doi.org/10.1016/j.infbeh.2021.101681}

\leavevmode\vadjust pre{\hypertarget{ref-Anis2020}{}}%
Anis, L., Perez, G., Benzies, K. M., Ewashen, C., Hart, M., \& Letourneau, N. (2020). Convergent validity of three measures of reflective function: Parent development interview, parental reflective function questionnaire, and reflective function questionnaire. \emph{Front Psychol}, \emph{11}, 574719. \url{https://doi.org/10.3389/fpsyg.2020.574719}

\leavevmode\vadjust pre{\hypertarget{ref-Arkle2023}{}}%
Arkle, P., Larkin, F., Wang, Y., Lee, Y., Fernandez, A., Li, L. Y., \& Meins, E. (2023). Early psychosocial risk factors and postnatal parental reflective functioning. \emph{Infancy}. \url{https://doi.org/10.1111/infa.12552}

\leavevmode\vadjust pre{\hypertarget{ref-Barreto2016}{}}%
Barreto, A. L., Fearon, R. M. P., Osório, A., Meins, E., \& Martins, C. (2016). Are adult mentalizing abilities associated with mind-mindedness? \emph{International Journal of Behavioral Development}, \emph{40}(4), 296--301. \url{https://doi.org/10.1177/0165025415616200}

\leavevmode\vadjust pre{\hypertarget{ref-Bjelland2002}{}}%
Bjelland, I., Dahl, A. A., Haug, T. T., \& Neckelmann, D. (2002). The validity of the hospital anxiety and depression scale. An updated literature review. \emph{J Psychosom Res}, \emph{52}(2), 69--77. \url{https://doi.org/10.1016/s0022-3999(01)00296-3}

\leavevmode\vadjust pre{\hypertarget{ref-Chao1994}{}}%
Chao, R. K. (1994). Beyond parental control and authoritarian parenting style: Understanding chinese parenting through the cultural notion of training. \emph{Child Dev}, \emph{65}(4), 1111--1119. \url{https://doi.org/10.1111/j.1467-8624.1994.tb00806.x}

\leavevmode\vadjust pre{\hypertarget{ref-Choi2004}{}}%
Choi, M., \& Harwood, J. (2004). A hypothesized model of korean women's responses to abuse. \emph{J Transcult Nurs}, \emph{15}(3), 207--216. \url{https://doi.org/10.1177/1043659604265115}

\leavevmode\vadjust pre{\hypertarget{ref-Choi2013}{}}%
Choi, Y., Kim, Y. S., Kim, S. Y., \& Park, I. K. (2013). Is asian american parenting controlling and harsh? Empirical testing of relationships between korean american and western parenting measures. \emph{Asian Am J Psychol}, \emph{4}(1), 19--29. \url{https://doi.org/10.1037/a0031220}

\leavevmode\vadjust pre{\hypertarget{ref-Costantini2017}{}}%
Costantini, A., Coppola, G., Fasolo, M., \& Cassibba, R. (2017). Preterm birth enhances the contribution of mothers' mind-mindedness to infants' expressive language development: A longitudinal investigation. \emph{Infant Behav Dev}, \emph{49}, 322--329. \url{https://doi.org/10.1016/j.infbeh.2017.10.006}

\leavevmode\vadjust pre{\hypertarget{ref-Dai2019a}{}}%
Dai, Q., Lim, A. K., \& Xu, Q. J. (2019). The relations between maternal mind-mindedness, parenting stress and obstetric history among chinese mothers. \emph{Early Child Development and Care}, \emph{189}(9), 1411--1424. \url{https://doi.org/10.1080/03004430.2017.1385608}

\leavevmode\vadjust pre{\hypertarget{ref-Dai2019b}{}}%
Dai, Q., McMahon, C., \& Lim, A. K. (2019). Cross-cultural comparison of maternal mind-mindedness among australian and chinese mothers. \emph{International Journal of Behavioral Development}, \emph{44}(4), 365--370. \url{https://doi.org/10.1177/0165025419874133}

\leavevmode\vadjust pre{\hypertarget{ref-Devine2019}{}}%
Devine, R. T., \& Hughes, C. (2019). Let's talk: Parents' mental talk (not mind-mindedness or mindreading capacity) predicts children's false belief understanding. \emph{Child Dev}, \emph{90}(4), 1236--1253. \url{https://doi.org/10.1111/cdev.12990}

\leavevmode\vadjust pre{\hypertarget{ref-Dollberg2022}{}}%
Dollberg, D. G. (2022). Mothers' parental mentalization, attachment dimensions and mother-infant relational patterns. \emph{Attach Hum Dev}, \emph{24}(2), 189--207. \url{https://doi.org/10.1080/14616734.2021.1901297}

\leavevmode\vadjust pre{\hypertarget{ref-Egeli2015}{}}%
Egeli, N. A., Rogers, W. T., Rinaldi, C. M., \& Cui, Y. (2015). Exploring the factor structure of the revised-parent as a social context questionnaire. \emph{Parenting}, \emph{15}(4), 269--287. \url{https://doi.org/10.1080/15295192.2015.1053334}

\leavevmode\vadjust pre{\hypertarget{ref-Fonagy1998b}{}}%
Fonagy, P., \& Target, M. (1998). Mentalization and the changing aims of child psychoanalysis. \emph{Psychoanalytic Dialogues}, \emph{8}(1), 87--114. \url{https://doi.org/10.1080/10481889809539235}

\leavevmode\vadjust pre{\hypertarget{ref-Fonagy1998a}{}}%
Fonagy, P., Target, M., Steele, H., \& Steele, M. (1998). \emph{Reflective-functioning manual, version 5.0, for application to adult attachment interviews}. Unpublished manuscript. London: University College London.

\leavevmode\vadjust pre{\hypertarget{ref-Fujita2021}{}}%
Fujita, N., \& Hughes, C. (2021). Mind-mindedness and self--other distinction: Contrasts between japanese and british mothers' speech samples. \emph{Social Development}, \emph{30}(1), 57--72. https://doi.org/\url{https://doi.org/10.1111/sode.12454}

\leavevmode\vadjust pre{\hypertarget{ref-Goldsmith1996}{}}%
Goldsmith, H. H., \& Rothbart, M. K. (1996). \emph{Prelocomotor and locomotor laboratory temperament assessment battery, lab-TAB; version 3.0}. Unpublished technical manual, Department of Psychology, University of Wisconsin, Madison, WI.

\leavevmode\vadjust pre{\hypertarget{ref-Harkness2000}{}}%
Harkness, S., Super, C. M., \& Tijen, N. van. (2000). Individualism and the "western mind" reconsidered: American and dutch parents' ethnotheories of the child. \emph{New Dir Child Adolesc Dev}, (87), 23--39. \url{https://doi.org/10.1002/cd.23220008704}

\leavevmode\vadjust pre{\hypertarget{ref-Hay2014}{}}%
Hay, D. F., Waters, C. S., Perra, O., Swift, N., Kairis, V., Phillips, R., \ldots{} Goozen, S. van. (2014). Precursors to aggression are evident by 6 months of age. \emph{Dev Sci}, \emph{17}(3), 471--480. \url{https://doi.org/10.1111/desc.12133}

\leavevmode\vadjust pre{\hypertarget{ref-Holloway2017}{}}%
Holloway, S. D. (2017). Teaching embodied: Cultural practice in japanese preschools by akiko hayashi and joseph tobin. \emph{The Journal of Japanese Studies}, \emph{43}(2), 499--504. \url{https://doi.org/10.1353/jjs.2017.0067}

\leavevmode\vadjust pre{\hypertarget{ref-Hong2016}{}}%
Hong, J. P., Lee, D. W., \& Ham, B. J. (2016). The epidemiological survey of mental disorders in korea. \emph{Sejong: Ministry of Health \& Welfare}, 15--18.

\leavevmode\vadjust pre{\hypertarget{ref-Hughes2018}{}}%
Hughes, C., Devine, R. T., \& Wang, Z. (2018). Does parental mind-mindedness account for cross-cultural differences in preschoolers' theory of mind? \emph{Child Dev}, \emph{89}(4), 1296--1310. \url{https://doi.org/10.1111/cdev.12746}

\leavevmode\vadjust pre{\hypertarget{ref-Jeong2011}{}}%
Jeong, G. Y., \& Shin, H. C. (2011). Validation of the korean version of parents as social context questionnaire (PSCQ). \emph{Korea Journal of Counselling}, \emph{12}(4), 1287--1305.

\leavevmode\vadjust pre{\hypertarget{ref-Jin2012}{}}%
Jin, M. K., Jacobvitz, D., Hazen, N., \& Jung, S. H. (2012). Maternal sensitivity and infant attachment security in korea: Cross-cultural validation of the strange situation. \emph{Attach Hum Dev}, \emph{14}(1), 33--44. \url{https://doi.org/10.1080/14616734.2012.636656}

\leavevmode\vadjust pre{\hypertarget{ref-Jones2013}{}}%
Jones, E., \& Coast, E. (2013). Social relationships and postpartum depression in south asia: A systematic review. \emph{Int J Soc Psychiatry}, \emph{59}(7), 690--700. \url{https://doi.org/10.1177/0020764012453675}

\leavevmode\vadjust pre{\hypertarget{ref-Keller2007}{}}%
Keller, H., Abels, M., Borke, J., Lamm, B., Su, Y., Wang, Y., \& Lo, W. (2007). Socialization environments of chinese and euro-american middle-class babies: Parenting behaviors, verbal discourses and ethnotheories. \emph{International Journal of Behavioral Development}, \emph{31}(3), 210--217. \url{https://doi.org/10.1177/0165025407074633}

\leavevmode\vadjust pre{\hypertarget{ref-Kim1994}{}}%
Kim, U., \& Choi, S.-H. (1994). \emph{Individualism, collectivism, and child development: A korean perspective} (P. M. Greenfield \& R. R. Cocking, Eds.). New York; London: Psychology Press.

\leavevmode\vadjust pre{\hypertarget{ref-Kim2005}{}}%
Kim, Uichol, Park, Y.-S., Kwon, Y.-E., \& Koo, J. (2005). Values of children, parent--child relationship, and social change in korea: Indigenous, cultural, and psychological analysis. \emph{Applied Psychology}, \emph{54}(3), 338--354. https://doi.org/\url{https://doi.org/10.1111/j.1464-0597.2005.00214.x}

\leavevmode\vadjust pre{\hypertarget{ref-Kirk2015}{}}%
Kirk, E., Pine, K., Wheatley, L., Howlett, N., Schulz, J., \& Fletcher, B. C. (2015). A longitudinal investigation of the relationship between maternal mind-mindedness and theory of mind. \emph{Br J Dev Psychol}, \emph{33}(4), 434--445. \url{https://doi.org/10.1111/bjdp.12104}

\leavevmode\vadjust pre{\hypertarget{ref-Krassner2017}{}}%
Krassner, A. M., Gartstein, M. A., Park, C., Dragan, W. Ł., Lecannelier, F., \& Putnam, S. P. (2017). East-west, collectivist-individualist: A cross-cultural examination of temperament in toddlers from chile, poland, south korea, and the u.s. \emph{Eur J Dev Psychol}, \emph{14}(4), 449--464. \url{https://doi.org/10.1080/17405629.2016.1236722}

\leavevmode\vadjust pre{\hypertarget{ref-Krink2021}{}}%
Krink, S., \& Ramsauer, B. (2021). Various mentalizing concepts in mothers with postpartum depression, comorbid anxiety, and personality disorders. \emph{Infant Ment Health J}, \emph{42}(4), 488--501. \url{https://doi.org/10.1002/imhj.21914}

\leavevmode\vadjust pre{\hypertarget{ref-Laranjo2010}{}}%
Laranjo, J., Bernier, A., Meins, E., \& Carlson, S. M. (2010). Early manifestations of children's theory of mind: The roles of maternal mind-mindedness and infant security of attachment. \emph{Infancy}, \emph{15}(3), 300--323. \url{https://doi.org/10.1111/j.1532-7078.2009.00014.x}

\leavevmode\vadjust pre{\hypertarget{ref-Larkin2021}{}}%
Larkin, F., Hayiou-Thomas, M. E., Arshad, Z., Leonard, M., Williams, F. J., Katseniou, N., \ldots{} Meins, E. (2021). Mind-mindedness and stress in parents of children with developmental disorders. \emph{J Autism Dev Disord}, \emph{51}(2), 600--612. \url{https://doi.org/10.1007/s10803-020-04570-9}

\leavevmode\vadjust pre{\hypertarget{ref-Larkin2019}{}}%
Larkin, F., Oostenbroek, J., Lee, Y., Hayward, E., \& Meins, E. (2019). Proof of concept of a smartphone app to support delivery of an intervention to facilitate mothers' mind-mindedness. \emph{PLoS One}, \emph{14}(8), e0220948. \url{https://doi.org/10.1371/journal.pone.0220948}

\leavevmode\vadjust pre{\hypertarget{ref-Lee2008b}{}}%
Lee, K. S., Chung, K. M., Park, J. A., \& Kim, H.-J. (2008). Reliability and validity study for the korean version of parenting stress index short form (k-PSI-SF). \emph{Korean Journal of Woman Psychology}, \emph{13}(3), 363--377. https://doi.org/\url{https://doi.org/10.18205/kpa.2008.13.3.007}

\leavevmode\vadjust pre{\hypertarget{ref-Lee2021}{}}%
Lee, Y., Meins, E., \& Larkin, F. (2021). Translation and preliminary validation of a korean version of the parental reflective functioning questionnaire. \emph{Infant Ment Health J}, \emph{42}(1), 47--59. \url{https://doi.org/10.1002/imhj.21883}

\leavevmode\vadjust pre{\hypertarget{ref-Lillard1998}{}}%
Lillard, A. (1998). Ethnopsychologies: Cultural variations in theories of mind. \emph{Psychol Bull}, \emph{123}(1), 3--32. \url{https://doi.org/10.1037/0033-2909.123.1.3}

\leavevmode\vadjust pre{\hypertarget{ref-Lundy2013}{}}%
Lundy, B. L. (2013). Paternal and maternal mind-mindedness and preschoolers' theory of mind: The mediating role of interactional attunement. \emph{Social Development}, \emph{22}(1), 58--74. https://doi.org/\url{https://doi.org/10.1111/sode.12009}

\leavevmode\vadjust pre{\hypertarget{ref-Luyten2017}{}}%
Luyten, P., Mayes, L. C., Nijssens, L., \& Fonagy, P. (2017). The parental reflective functioning questionnaire: Development and preliminary validation. \emph{PLoS One}, \emph{12}(5), e0176218. \url{https://doi.org/10.1371/journal.pone.0176218}

\leavevmode\vadjust pre{\hypertarget{ref-McMahon2017}{}}%
McMahon, C. A., \& Bernier, A. (2017). Twenty years of research on parental mind-mindedness: Empirical findings, theoretical and methodological challenges, and new directions. \emph{Developmental Review}, \emph{46}, 54--80. \url{https://doi.org/10.1016/j.dr.2017.07.001}

\leavevmode\vadjust pre{\hypertarget{ref-McMahon2012}{}}%
McMahon, C. A., \& Meins, E. (2012). Mind-mindedness, parenting stress, and emotional availability in mothers of preschoolers. \emph{Early Childhood Research Quarterly}, \emph{27}(2), 245--252. https://doi.org/\url{https://doi.org/10.1016/j.ecresq.2011.08.002}

\leavevmode\vadjust pre{\hypertarget{ref-Meins1997}{}}%
Meins, E. (1997). \emph{Security of attachment and the social development of cognition}. Hove: Lawrence Erlbaum Associates.

\leavevmode\vadjust pre{\hypertarget{ref-Meins2013}{}}%
Meins, E., Centifanti, L. C. M., Fernyhough, C., \& Fishburn, S. (2013). Maternal mind-mindedness and children's behavioral difficulties: Mitigating the impact of low socioeconomic status. \emph{J Abnorm Child Psychol}, \emph{41}(4), 543--553. \url{https://doi.org/10.1007/s10802-012-9699-3}

\leavevmode\vadjust pre{\hypertarget{ref-Meins2015}{}}%
Meins, E., \& Fernyhough, C. (2015). \emph{Mind-mindedness coding manual, version 2.2}. Unpublished manuscript. University of York, York, UK.

\leavevmode\vadjust pre{\hypertarget{ref-Meins2019}{}}%
Meins, E., Fernyhough, C., \& Centifanti, L. C. M. (2019). Mothers' early mind-mindedness predicts educational attainment in socially and economically disadvantaged british children. \emph{Child Dev}, \emph{90}(4), e454--e467. \url{https://doi.org/10.1111/cdev.13028}

\leavevmode\vadjust pre{\hypertarget{ref-Meins2001}{}}%
Meins, E., Fernyhough, C., Fradley, E., \& Tuckey, M. (2001). Rethinking maternal sensitivity: Mothers' comments on infants' mental processes predict security of attachment at 12 months. \emph{The Journal of Child Psychology and Psychiatry and Allied Disciplines}, \emph{42}(5), 637--648. \url{https://doi.org/10.1017/S0021963001007302}

\leavevmode\vadjust pre{\hypertarget{ref-Meins2012}{}}%
Meins, E., Fernyhough, C., Rosnay, M. de, Arnott, B., Leekam, S. R., \& Turner, M. (2012). Mind-mindedness as a multidimensional construct: Appropriate and nonattuned mind-related comments independently predict infant-mother attachment in a socially diverse sample. \emph{Infancy}, \emph{17}(4), 393--415. \url{https://doi.org/10.1111/j.1532-7078.2011.00087.x}

\leavevmode\vadjust pre{\hypertarget{ref-Meins1998}{}}%
Meins, E., Fernyhough, C., Russell, J., \& Clark-Carter, D. (1998). Security of attachment as a predictor of symbolic and mentalising abilities: A longitudinal study. \emph{Social Development}, \emph{7}(1), 1--24. https://doi.org/\url{https://doi.org/10.1111/1467-9507.00047}

\leavevmode\vadjust pre{\hypertarget{ref-Meins2003}{}}%
Meins, E., Fernyhough, C., Wainwright, R., Clark-Carter, D., Das Gupta, M., Fradley, E., \& Tuckey, M. (2003). Pathways to understanding mind: Construct validity and predictive validity of maternal mind-mindedness. \emph{Child Dev}, \emph{74}(4), 1194--1211. \url{https://doi.org/10.1111/1467-8624.00601}

\leavevmode\vadjust pre{\hypertarget{ref-Meins2002}{}}%
Meins, E., Fernyhough, C., Wainwright, R., Das Gupta, M., Fradley, E., \& Tuckey, M. (2002). Maternal mind-mindedness and attachment security as predictors of theory of mind understanding. \emph{Child Dev}, \emph{73}(6), 1715--1726. \url{https://doi.org/10.1111/1467-8624.00501}

\leavevmode\vadjust pre{\hypertarget{ref-Miller1998}{}}%
Miller, A. L., \& Sameroff, A. J. (1998). \emph{Mother-infant coding system (7 months): Michigan family study}.

\leavevmode\vadjust pre{\hypertarget{ref-Nijssens2018}{}}%
Nijssens, L., Bleys, D., Casalin, S., Vliegen, N., \& Luyten, P. (2018). Parental attachment dimensions and parenting stress: The mediating role of parental reflective functioning. \emph{Journal of Child and Family Studies}, \emph{27}(6), 2025--2036. \url{https://doi.org/10.1007/s10826-018-1029-0}

\leavevmode\vadjust pre{\hypertarget{ref-OBrien2014}{}}%
O'Brien, K. M., Ganginis Del Pino, H. V., Yoo, S.-K., Cinamon, R. G., \& Han, Y.-J. (2014). Work, family, support, and depression: Employed mothers in israel, korea, and the united states. \emph{J Couns Psychol}, \emph{61}(3), 461--472. \url{https://doi.org/10.1037/a0036339}

\leavevmode\vadjust pre{\hypertarget{ref-Oh1999}{}}%
Oh, S. M., Min, K. J., \& Park, D. B. (1999). A comparison of normal, depressed and anxious groups: A study on the standardization of the hospital anxiety and depressed scale for koreans. \emph{Journal of Korean Neuropsychiatic Association}, \emph{38}(2), 289--296.

\leavevmode\vadjust pre{\hypertarget{ref-Pajulo2018}{}}%
Pajulo, M., Tolvanen, M., Pyykkönen, N., Karlsson, L., Mayes, L., \& Karlsson, H. (2018). Exploring parental mentalization in postnatal phase with a self-report questionnaire (PRFQ): Factor structure, gender differences and association with sociodemographic factors. The finn brain birth cohort study. \emph{Psychiatry Res}, \emph{262}, 431--439. \url{https://doi.org/10.1016/j.psychres.2017.09.020}

\leavevmode\vadjust pre{\hypertarget{ref-Park2006}{}}%
Park, Y.-S., \& Kim, U. (2006). Family, parent-child relationship, and academic achievement in korea. In Uichol Kim, K.-S. Yang, \& K.-K. Hwang (Eds.), \emph{Indigenous and cultural psychology: Understanding people in context} (pp. 421--443). Boston, MA: Springer US. \url{https://doi.org/10.1007/0-387-28662-4_19}

\leavevmode\vadjust pre{\hypertarget{ref-Rostad2016}{}}%
Rostad, W. L., \& Whitaker, D. J. (2016). The association between reflective functioning and parent--child relationship quality. \emph{Journal of Child and Family Studies}, \emph{25}(7), 2164--2177. \url{https://doi.org/10.1007/s10826-016-0388-7}

\leavevmode\vadjust pre{\hypertarget{ref-Rutherford2013}{}}%
Rutherford, H. J. V., Goldberg, B., Luyten, P., Bridgett, D. J., \& Mayes, L. C. (2013). Parental reflective functioning is associated with tolerance of infant distress but not general distress: Evidence for a specific relationship using a simulated baby paradigm. \emph{Infant Behav Dev}, \emph{36}(4), 635--641. \url{https://doi.org/10.1016/j.infbeh.2013.06.008}

\leavevmode\vadjust pre{\hypertarget{ref-Schacht2017}{}}%
Schacht, R., Meins, E., Fernyhough, C., Centifanti, L. C. M., Bureau, J.-F., \& Pawlby, S. (2017). Proof of concept of a mind-mindedness intervention for mothers hospitalized for severe mental illness. \emph{Dev Psychopathol}, \emph{29}(2), 555--564. \url{https://doi.org/10.1017/S0954579417000177}

\leavevmode\vadjust pre{\hypertarget{ref-Sharp2008}{}}%
Sharp, C., \& Fonagy, P. (2008). The parent's capacity to treat the child as a psychological agent: Constructs, measures and implications for developmental psychopathology. \emph{Social Development}, \emph{17}(3), 737--754. https://doi.org/\url{https://doi.org/10.1111/j.1467-9507.2007.00457.x}

\leavevmode\vadjust pre{\hypertarget{ref-Skinner2005}{}}%
Skinner, E., Johnson, S., \& Snyder, T. (2005). Six dimensions of parenting: A motivational model. \emph{Parenting}, \emph{5}(2), 175--235. \url{https://doi.org/10.1207/s15327922par0502_3}

\leavevmode\vadjust pre{\hypertarget{ref-Slade2005}{}}%
Slade, A. (2005). Parental reflective functioning: An introduction. \emph{Attach Hum Dev}, \emph{7}(3), 269--281. \url{https://doi.org/10.1080/14616730500245906}

\leavevmode\vadjust pre{\hypertarget{ref-Steele2020}{}}%
Steele, K. R., Townsend, M. L., \& Grenyer, B. F. S. (2020). Parenting stress and competence in borderline personality disorder is associated with mental health, trauma history, attachment and reflective capacity. \emph{Borderline Personal Disord Emot Dysregul}, \emph{7}, 8. \url{https://doi.org/10.1186/s40479-020-00124-8}

\leavevmode\vadjust pre{\hypertarget{ref-Zigmond1983}{}}%
Zigmond, A. S., \& Snaith, R. P. (1983). The hospital anxiety and depression scale. \emph{Acta Psychiatr Scand}, \emph{67}(6), 361--370. \url{https://doi.org/10.1111/j.1600-0447.1983.tb09716.x}

\end{CSLReferences}

\newpage

\hypertarget{supplementary-materials}{%
\section*{Supplementary Materials}\label{supplementary-materials}}
\addcontentsline{toc}{section}{Supplementary Materials}

\begin{table}[H]

\caption{\label{tab:table1}Descriptive Statistics and Cultural Differences}
\centering
\fontsize{8}{10}\selectfont
\begin{tabular}[t]{lcccc}
\toprule
\textbf{ } & \textbf{United Kingdom M(SD)} & \textbf{South Korea M(SD)} & \textbf{Cultural Comparison} & \textbf{p}\\
\midrule
HADS & 11.05 (5.12) & 13.42 (6.49) & t(126) = 2.29 & 0.024\\
PSI-PD & 27.32 (7.34) & 32.52 (8.25) & t(126) = 3.75 & <0.001\\
PSI-CDI & 19.16 (5.58) & 18.71 (5.81) & t(126) = -0.45 & 0.657\\
PSI-DC & 23.27 (7.2) & 23.24 (7.78) & t(126) = -0.02 & 0.981\\
Positive Parenting & 53.03 (3.7) & 49.85 (5.32) & t(127) = -3.93 & <0.001\\
\addlinespace
Negative Parenting & 25.25 (5.78) & 26.58 (5.96) & t(127) = 1.28 & 0.204\\
Observed Intrusiveness & 0.83 (0.59) & 1.28 (0.72) & t(127) = 3.9 & <0.001\\
Observed Infant Temp. & 2.93 (2.01) & 3.56 (2.61) & t(127) = 1.54 & 0.125\\
\bottomrule
\end{tabular}
\end{table}

\emph{Note.} HADS = The Hospital Anxiety and Depression Scale; PSI-PD = Parenting Stress Index-Short Form Parenting distress subscale; PSI-CDI = Parenting Stress Index-Short Form Parent--child dysfunctional interaction subscale; PSI-DC = Parenting Stress Index-Short Form Difficult child subscale.

\newpage

\begin{table}[H]

\caption{\label{tab:table2}Descriptive Statistics and Cultural Differences in Parental Mentalization}
\centering
\begin{tabular}[t]{lccc}
\toprule
\textbf{ } & \textbf{Mind Related Comments} & \textbf{United Kingdom M(SD)} & \textbf{South Korea M(SD)}\\
\midrule
AMRC(proportion) & Total AMRC & 0.06 (0.03) & 0.06 (0.04)\\
 & Desire and preferences & 0.74 (0.22) & 0.58 (0.29)\\
 & Cognition & 0.12 (0.17) & 0.21 (0.23)\\
 & Intention & 0.04 (0.09) & 0.01 (0.05)\\
 & Emotion & 0.07 (0.13) & 0.15 (0.22)\\
\addlinespace
NMRC(proportion) & Total NMRC & 0.02 (0.02) & 0.01 (0.01)\\
 & Desire and preferences & 0.68 (0.4) & 0.37 (0.41)\\
 & Cognition & 0.06 (0.17) & 0.08 (0.21)\\
 & Intention & 0.01 (0.06) & 0.02 (0.11)\\
 & Emotion & 0.03 (0.14) & 0.08 (0.17)\\
\addlinespace
 & Physical statement & 0.03 (0.1) & 0.15 (0.29)\\
PRFQ & CMS & 3.94 (0.88) & 4.88 (0.88)\\
 & IC & 6.09 (0.65) & 6 (0.66)\\
\bottomrule
\end{tabular}
\end{table}

\emph{Note.} AMRC = Appropriate Mind-Related Comments; NMRC = Non-attuned Mind-Related Comments; PRFQ = Parental Reflective Functioning Questionnaire; CMS = Certainty about Mental States; IC = Interest and Curiosity in Mental states.

\newpage

\begin{table}[H]

\caption{\label{tab:Table3prep}Associations between Maternal Mind-mindedness and the PRFQ Subscales}
\centering
\fontsize{9}{11}\selectfont
\begin{tabular}[t]{llcccc}
\toprule
\textbf{ } & \textbf{  } & \textbf{AMRC Pearson's r} & \textbf{AMRC Bayes Factor} & \textbf{NMRC Pearson's r} & \textbf{NMRC Bayes Factor}\\
\midrule
GB & CMS & 0.004 & 10.09 & -0.03 & 9.84\\
 & IC & 0.064 & 8.93 & 0.11 & 6.8\\
KO & CMS & 0.137 & 5.69 & 0.18 & 3.82\\
 & IC & 0.179 & 3.72 & 0.2 & 2.88\\
\bottomrule
\end{tabular}
\end{table}

\emph{Note.} AMRC = Appropriate Mind-Related Comments; NMRC = Non-attuned Mind-Related Comments; CMS = Certainty about mental states; IC = Interest and curiosity in mental states.

\end{document}
